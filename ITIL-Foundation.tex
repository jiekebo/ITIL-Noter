%%%--- TITEL OG FORFATTER ---%%%
\newcommand{\tit}{ITIL Foundation - Noter}
\newcommand{\aut}{Jacob Salomonsen}
\newcommand{\subj}{Steria}
\newcommand{\mail}{\href{mailto:jsa@steria.dk}{jsa@steria.dk}}

%%%--- PREAMBLE INCLUDE ---%%%
\input{Preamble.tex}
%\usepackage[danish]{babel}				% Dansk orddeling osv.
%\usepackage[fixlanguage]{babelbib}		% Sprogpakke til BibTeX
%\selectbiblanguage{danish}				% Sprogvalg til BibTeX

%%%--- LISTISTINGS SETUP ---%%%
\lstset{ %
language=ML,							% choose the language of the code
basicstyle=\footnotesize,				% the size of the fonts that are used for the code
numbers=left,							% where to put the line-numbers
numberstyle=\footnotesize,				% the size of the fonts that are used for the line-numbers
stepnumber=2,							% the step between two line-numbers. If it's 1 each line will be numbered
numbersep=5pt,							% how far the line-numbers are from the code
backgroundcolor=\color{white},			% choose the background color. You must add \usepackage{color}
showspaces=false,						% show spaces adding particular underscores
showstringspaces=false,					% underline spaces within strings
showtabs=false,							% show tabs within strings adding particular underscores
frame=single,							% adds a frame around the code
tabsize=2,								% sets default tabsize to 2 spaces
captionpos=b,							% sets the caption-position to bottom
breaklines=true,						% sets automatic line breaking
breakatwhitespace=false,				% sets if automatic breaks should only happen at whitespace
escapeinside={\%*}{*)}					% if you want to add a comment within your code
}

%%%--- DOKUMENT STARTER HER ---%%%
\begin{document}

%%%--- Title ---%%%%
\begin{titlepage}
\HRule
\begin{center}\huge{\bfseries{\tit}}\end{center}
\HRule
%\\[2.0cm]
\begin{center}
\aut
\\[0.5cm]
\mail
\\[0.5cm]
\subj
\end{center}
\vfill
\begin{center}\today\end{center}
\end{titlepage}

%%%--- Indhold ---%%%
%\tableofcontents
%\newpage

%%%--- Tekst ---%%%

\section{Intro}
ITIL handler om kunders forventining til service. ITIL består af 5 faser der varetager en applikations livscyklus:

\begin{itemize}
\item Continuous improvement
\item Operation
\item Transition
\item Design
\item Strategy
\end{itemize}

Produkt kontra service: Ved service er det leverandørens ansvar at sørge for kvaliteten. Modsat ved et produkt (ex. mælk i Netto) er det kundens ansvar at sørge for at kvaliteten lever op til forventning.

ITIL søger at klarlægge behovet. Forretningsprocesser som skal understøttes.

Alle services kan granuleres til underlinggende services og produkter, inklusiv hardware og software.

SLA: kontrakt mellem service leverandør om kunde. Kunden ser kun selve service.

\section{Hvad er en service}

Ved CSM løsning skal man kunne ringe ud og ind. Sende og modtage mail/SMS.
\begin{itemize}
\item Utility: Hvad kan servicen rent funktionelt.
\item Warranty: Tilgænglighed, kapacitet, sikkerhed, kontinuitet.
\end{itemize}

SLA består af utility og warranty som er målbart. Alt der ikke kan måles SKAL ikke inkluderes i SLA.

Forretningsproces: Det kunderne gør som de har brug for hjælp til fra os.

ITIL service management består af:
\begin{center}
\begin{tabular}{l||c|c|c|c}
kunder & SKAT & ASK & Top DK. & DEAS \\
Forretnings process. & Toldsystemer & Kundeservice & Administration & \\
Ansvaret skiller her & SLA & SLA & SLA & SLA\\
IT-Services & CSM & Desktop mgmt. & BPM & Faktura scanning \\
ITIL er i dette lag & & &  & \\
Komponenter & HW & APP & Netv. & Data
\end{tabular}
\end{center}


\section{Service Management}
\subsection{ITIL som rammeværk}
ISO 20000 er standard for it service management. ITIL er et rammeværk hvilket vil sige det kan tilpasses til virksomhedens virkelighed.
Itil er en løbende process som udvikles kontinuert. ITIL versioner er forskellige fra hindanden.

\begin{itemize}
\item Utility
\item Warranty
\end{itemize}

Utility består af performance eller fjernelse af begrænsninger.

De fire warranty elementer
\begin{itemize}
\item Availability (Alle relationer mellem en service skal undersøges i forbindelse med avaliability).
\item Capacity (Svartider, capacity påvirker availability)
\item Continuity (Katastrofeberedskab, back-up planer)
\item Security (Beskytning af data)
\end{itemize}

Alle elementer kan måles og sættes ind i en SLA

Organisatoriske capabilities: evner i virksomheden, personers reele funktioner.

It service som består af utility og warranty. Kunden ser oftest mest på utility.

People, Processes, Products, Partners

Kunden har selv ansvar for at deres forretningsprocesser udføres af servicen, som it-organisation har vi kune indflydelse på dette. Jo mere man ved om hvad forretningen vil opnå jo bedre er slutproduktet (outcome).

\subsection{Iteressenter}
Funktion har en særlig betydning i ITIL: Et team eller en gruppe af medarbejder som har værktøjer til rådighed eller andre ressourcer de anvender for at udføre processer eller aktiviteteter.

\subsection{Assets}
Assets er alt det man bruger til at levere en service. Opdeles i capabilities og ressourcer. 

Alle ressourcer kan måles på en eller anden facon.

Capabilities er ikke håndgribelige som Ressourcer. Processer i diagrammet, evnen til at følge processer. Capabilities styrer og koordinerer ressourcer gennem processer.

Sammenhængen mellem capability og ressourcer, er den evne man har til at udnytte resourcer.

\subsection{Procesmodellen}
Der er altid noget der trigger en proces.

Så snart man har klarlagt/tegnet processen vil man kunne finde målepunkter.

Formidlere (enablers), gør processen mulig. Processkontrol styrer processen og sørger for at den følges.

Feedback kan komme i form af forbedringsønsker eller feedback på målinger.

Incident producerer workaround og kan trigger problem.

Change producerer en behandlet change request (ja/nej).

Der kan være mange servicejere men der kan kun være en ejer per service.

Silodannelse, en afdeling afsondrer sig fra resten af organisationen, højt specialiseret. ITIL higer efter at nedbryde siloer.

Accountable: man kan stilles til regnskab for det arbejde der bliver udført.

Der eksisterer mange til mange forhold mellem services og processer.

Service desk eller team lead vil ofte være process manager, som sørger for at service desken har sin normale daglige operation. Kan sammelignes med en værkfører.

Process ejeren har ikke indblik i den daglige operation, står uden for processen, men definerer selve process med målepunkter.

Matriceorganisation dækker over at man ikke skaber siloer.

\subsection{Automatisering}
Hvad skal værktøj dække af formål. Man skal kende processen før man kan finde værktøj. 

Change hører til i transitions værktøj da den er en del af udrulning og sætter ting i gang.

OLA intern SLA.

\section{Service Strategy}
Tag udgangspunkt i forretningens slutprodukt, resultat, outcome, udbytte.
Denne fase handler om at finde frem til hvad kunden skal bruge.

Virksomhedsstrategi, et overordnet mål for hele virksomheden.

IT-Strategi, hvilke tekniske mål sættes for organisationen. Eksempel, brug MS produkter.

Service strategi bliver påvirket af alle strategier.

Værdi er ikke altid at definere, da kunden er med til at definere store dele af dette. Man kan definere Utility og Warranty.
Kender man kundes preferencer kan man få utility og warranty til at betyde mindre. 

Tre forskellige typer services

\begin{itemize}
\item Core services (Basale services som service normalt indeholder, direkte understøtter forretningsprocessen)(Klipning)
\item Enabling services (services som kunden sjældent ser)(slipning af saks)
\item Enhancing services (Klare services som ses som ekstra der giver en wow faktor)(Ugeblade, kaffe)(bliver efter tid til core services)
\end{itemize}

\subsection{Serviceleverandør typer}
Intern serviceleverandør, leverer til specifik afdeling.

Shared service unit, leverer til flere afdelinger.

Ekstern serviceleverandør, leverer til kunde, outsourcing.

\subsection{Governance}
Kontrol og beslutningstagen. Jo mere kontrol og styring jo sværere bliver det at være effektiv og producere. Der skal skabes en balance mellem kontrol styring kontra frihed og handlemulighed. Regler overfor beslutninger.

Orderliness -- Performance.

Udlevelse af strategier og retningslinier med fokus på balance mellem ovenstående, uden at give for meget køb på performance.

ITIL fokuserer mest på værdiskabelse og ressourceudnyttelse.

\subsection{Risk management}
Risk management er en bred disciplin som benyttes i alle faser af ITIL.

\begin{itemize}
\item identificere
\item analysere
\item håndter
\end{itemize}

Vurdering af sandsynlighed og impact. Hvad koster counter-measures. Dette giver en prioritering af hvilke risici der skal håndteres først.

\subsection{Service portfolio management}
At sørge for at have de services der løser kundens behov.

Hvis kundernes behov flytter sig skal spm sørge for at services også flytter sig.

I et CMS findes information om service portfolio på alle niveauer.

Strategisk investering i leverance apparatet, er investeringer organisationen gør som kunden ikke er involveret i, evt en nødvendig opgradering.

Kommisorium, der skal være kunder som kan levere penge til en service.

Portfolio management:
\begin{itemize}
\item Definerer
\item Analyserer
\item Godkender (Business case opstår her)
\item Charter (penge)
\end{itemize}

Leder til service design fasen:
\begin{itemize}
\item Udvikling (Service design)
\item Idriftsættelse (Service transition)
\item Drift (Service operation)(Service catalog)
\item Udfasning (Service transition) (Portfolion management udføres også her)
\end{itemize}

Som til sidst leder til udfasning af service.

\subsection{Financial management}

\subsection{Business case}
Cost benefit vurdering af krav.


\subsection{Business Relationship management}
Strategisk: mulighed for mersalg, både eksternt til kunder men også til internt brug. Dette giver øget mersalg og effektivitet internt men viser sig også eksternt. Vær tæt på forretningen. (business relationship manager med produkt katalog)

Taktisk: er mere en SDM, ikke ITIL, Service Level manager. service ejer. Håndtere SLA'er med kunden.

Operationelt: Service desk, direkte kontakt.

Product owner er tættest på service ejer i ITIL.

Kundens strategiske del kan være langt fra selve brugerne.

\subsection{Demand management}
Der kan forekomme udsving i kundernes behov. Dette kan enten koste at dække da der uden for spidsbelastning kan opstå spild. Man kan dimensionere service efter normal kørsel, men dette leder til utilfredse kunder.

I stedet for ovenstående kan man motivere kunder til at modificere deres adfærd ved enten pisk eller gulerod.

Demand management, styring af efterspørgslen
Capacity management, styring af selve kapaciteten.


\subsection{Opsummering}

Kunde og IT ledelse aftaler SLA. IT Ledelse sætter udvikling i gang. Udvikling afleverer til brugere. Brugerne melder krav og behov til kunden. Hvis udvikling kommunikerer med brugere kan det give bedre løsning. Forventningens vej.

Udbud -> Service katalog.

Efterspørgsel -> Service pipeline.

Outcome


\section{Service Design}
Hvordan får vi lavet hvad kunden har behov for. Overordnede ting som peger i retning af hvad kunde har brug for. I designfasen gælder det om ikke at overlade noget til tilfældigheder. Derfor er det et spørgsmål om at være opmærksom på alle detaljer.

Man ser på service landskabet først. Man gør det i den rigtige rækkefølge, alltså fra kundens synspunkt.

Design fasen indeholde også design af selve processerne.

I design fasen laves der kontrakter på samme måde som i service strategy laves SLA.

De 4 p'er (People, processes, product, partners) skal indgå i designfasen.

Designer politikker og retningslinier for service, se process for itil.

Hovedaspekter af service design
\begin{itemize}
\item Serviceløsninger (Holistisk se overordnet på outcome)
\item Management information systemer - portfolio (Portfolio)
\item Teknologi (server/klient, data arkitektur, produkt arkitektur, monitorering)
\item Processer (Skal nuværende processer udvides, kun ITIL)
\item Målesystemer (SLA og OLA skal der måles efter om opfyldes)
\end{itemize}

\subsection{Service løsninger}
Early life support opstår når en service overgår fra transition til operation. Her omformuleres SLA hvis service ikke lever op til kravene. Service level management.


\subsection{Teknologi og management arkitektur}
Service arkitektur går ud på at anskueliggøre over for kunden, de offerings som leverandøren kan give. Ikke andet end service arkitektur bør være synlig for kunden. Dette gør at kunden ikke har indflydelse på selve applikationens beskaffenhed.

\subsection{Målesystemer}
\begin{itemize}
\item Teknologimetrikker: Det meste teknologi har målesystemer indbygget.
\item Servicemetrikker: SLA målinger, tilgængelighed, kundens tilfredshed, leveringstider på ny pc. Set fra kundens side. 
\item Processmetrikker: Måling på selve processen, incidentprocess, change process, løsningstid, svartid, eskaleringer.
\item Kundemetrikker: Kundeklager etc.
\item IT mål og metrikker: Budget der skal overholdes, eller indtjæning.
\item Forretningsmål og metrikker: Lever service op til virksomhedens overordnede mål
\end{itemize}


\subsection{RACI matricen}
En RACI matrice definerer ansvar i en process. Kolonner er roller, rækker er aktiviteter i processen.
\begin{itemize}
\item Accountable - Man har det overordnede ansvar for at ting tager sted.
\item Responsible -  Dem der skal udføre selve opgaverne. (Proces praktikere)
\item Consulted - Man bør stå til rådighed for konsultering om sagen.
\item Informed - Modtager kun informationer, man håndterer afvigelser i denne rolle.
\end{itemize}

\subsection{kompetencer og evner}
SFIA, skills framework for the information age kan bruges til at definere capabilities i en organisation.

\subsection{Service design processer}
Der designes en service med fokus på warranty elementer, som der findes processer for at designe.

\subsubsection{Service level management}
Kunden der definerer, burde være os.

\subsubsection{Capacity management}


\subsubsection{Information security management}

\subsubsection{Supplier management}
Leverandør kontrakter.

\subsubsection{Design coordination}
Sikrer at hvert service design fasen går i den rigtige retning. Hvordan er designpolitiken, planlægge på tværs af afdelinger (portfolio management). For hvert enkelt design koordinerer man individuelle design til overlevering til transition fasen.

\subsubsection{Service catlog management}
Service portfolio management er den overordnede process, som styrer servcies i pipeline og i udfasning. SPM leverer input til service catalog management som består i at beskrive services der er i produktion. Skal vedligeholde en kilde til eksisterende og kommende operationelle services. Man kan godt inkludere servcies i udfasning. Denne fase har stor grænseflade med service level management.

Configuration items er det samme som service assets. SCM skal sikrer at grænseflader ned til CI er definerede. Alle komponenter bliver benævnt configuration items.

For hver relation mellem kunde og service i service kataloget er der en SLA.

Service katalog: Set fra services udbudt kunder.

Teknisk service katalog: Set fra de enkelte komponenter.

\subsubsection{Service level management}
Aftale mellem kunde og it-leverandør -> SLA. SLA er en gensidig aftale. Udarbejdelse af aftaler og sørge for at aftalerne bliver overholdt. 

Forventningsafstemning via kundemøder. Proaktive handliner der sørger for at tilfredsstille kundens behov.

Aftaler med kunden skrives i kundesprog.

OLA laves internt i it-organisationen, som skal understøtte at SLA kan gennemføres. OLA er typisk skrevet i teknisk/internt sprog.

Taktisk niveau med business relationship management.

Service level requirements, de service krav som kunden måtte have, forud for SLA'er. SLR bliver konverteret til Service level targets (SLT) som skrives ind i SLA efter evt. modifikation.

Der måles på SLA'er og evt. forbedringer iværksættes.

SLA har til formål at forbedre service. Definition af mål som burde kunne forandres.

Kontrakt har til formål at definere klare mål som skal overholdes. Jura.

SLA og kontrakt bør ikke blandes sammen.

Mange interfaces til forskellig dele af ITIL. Supplier management, styr på kontrakter mellem it-org. og kunde.

\begin{itemize}
\item Service baseret SLA - een sla man giver til alle kunder. Sparer vedligeholdelse.
\item Kundebaseret SLA - En sla som gælder for alle services en kunde har. Sparer vedligeholdelse.
\item Multilevel SLA - Introducerer hierarki i SLA, som gør at man kan abstrahere i SLA. Man kan på den måde skabe en abstrakt SLA som gælder corporate, hvor specifikke dele kan implementeres. (Corp, Customer, Service)
\end{itemize}

\subsubsection{}
























\section{Forkortelser / beskrivelser}
\begin{itemize}
\item OLA - Operational Level Agreement - Intern aftale som skal støtte op om overholdelse af SLA. Procent oppetid på server, tabte netværkspakker.
\item SLA - Service Level Agreement - Kontrakt mellem service leverandør og kunde. Utility/Warranty. Svartid på kundehendvendelser, benahdlingstid på ændringsønsker. Kan ændre sig over tid. En samling af SLT'er som er konverteret fra SLR'er.
\item SAC - Service Acceptance Criteria - Overordnede kriterier som en service skal kunne opfylde i hele levetiden.
\item SLR - Service Level Requirement - Kundens krav til en service der ligger forud for SLA i transitionsfasen.
\item SDP - Service Design Package - Den "perfekte" beskrivelse af hvordan service skal implementeres. Perfekt blueprint. Fuldstændig specifikation af alle parametre for service.
\item SLT - Service level target - Del af en SLA, et enkelt service mål som kommer af en SLR.
\item RFC - Request For Change
\item Best Practice
\item Standard - En process man skal følge slavisk
\item Rammeværk - En definition af best practices som man kan vælge at følge.
\item SFIA - skills framework for the information age
\item SLAM (Chart) - Oversigt over SLA og deres overholdelse
\end{itemize}





%%%---BibTeX ---%%%
%(Compile: 1 x LaTeX 1 x Bibtex 2 x LaTex)%
%\newpage
%\bibliographystyle{plain}
%\bibliography{/Tex/Library}

\end{document}
