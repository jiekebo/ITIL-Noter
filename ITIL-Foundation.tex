%%%--- TITEL OG FORFATTER ---%%%
\newcommand{\tit}{ITIL Foundation - Noter}
\newcommand{\aut}{Jacob Salomonsen}
\newcommand{\subj}{Steria}
\newcommand{\mail}{\href{mailto:jsa@steria.dk}{jsa@steria.dk}}

%%%--- PREAMBLE INCLUDE ---%%%
%%%--- Preamble by Jacob Salomonsen ---%%%
\documentclass[a4paper,10pt]{article}

%%%--- IMPORT AF PAKKER ---%%%
\usepackage[hang,small,bf]{caption}		% Nice captions at figures
\usepackage[utf8]{inputenc}			% Input encoding
\usepackage[T1]{fontenc}				% Font encoding
\usepackage{amsmath}					% Matematiske kommandoer
\usepackage{amssymb}					% Matematiske symboler
\usepackage{amsfonts}					% Matematiske fonte
\usepackage{listings}					% Listing af kildekode
\usepackage{color}						% Color package used in listings
\usepackage{fancyhdr}					% Fancy header
\usepackage{booktabs}					% Flotte tabeller
\usepackage{color}						% Farver på listings
\usepackage{subfigure}					% Figurer i linje
%\usepackage{subfig}					% Figurer i linje
\usepackage{float}						% Customized floats
\usepackage{wrapfig}					% Tekst omkring figurer
\usepackage{tikz}							% Tikz tegning
\usetikzlibrary{automata,shadows,arrows,decorations.pathmorphing,backgrounds,positioning,fit}
\usepackage{pgf}							% Pgf tegning
\usepackage{hyperref}					% Hyper-links i PDF

%%%--- SKRIFTTYPE ---%%%
%\usepackage{stmaryrd}
%\usepackage{mathpazo}
\usepackage{charter}
%\usepackage{helvet}
%\usepackage{avant}
%\usepackage{mathptmx}
%\usepackage{newcent}
%\usepackage{chancery}
%\usepackage{bookman}
%\usepackage{pifont}
%\usepackage{fourier}
%\usepackage{courier}

%%%--- HYPERREF SETUP ---%%%
\hypersetup{
    %bookmarks=true,					% show bookmarks bar?
    unicode=false,						% non-Latin characters in Acrobat's bookmarks
    pdftoolbar=true,					% show Acrobat's toolbar?
    pdfmenubar=true,					% show Acrobat's menu?
    pdffitwindow=false,					% window fit to page when opened
    pdfstartview={FitH},				% fits the width of the page to the window
    pdftitle={\tit},					% title
    pdfauthor={\aut},					% author
    pdfsubject={Subject},				% subject of the document
    pdfcreator={Creator},				% creator of the document
    pdfproducer={Producer},				% producer of the document
    pdfkeywords={keywords},				% list of keywords
    pdfnewwindow=true,					% links in new window
    colorlinks=true,					% false: boxed links; true: colored links
    linkcolor=blue,						% color of internal links
    citecolor=blue,						% color of links to bibliography
    filecolor=blue,						% color of file links
    urlcolor=blue						% color of external links
}

%%%--- 1,5 LINJEAFSTAND ---%%%
%\linespread{1,3}

%%%--- SIDE LAYOUT ---%%%
\addtolength{\textwidth}{2.0cm}
\addtolength{\evensidemargin}{1.0cm}
\addtolength{\oddsidemargin}{-1.0cm}
\setlength{\parindent}{0.0cm}
\setlength{\headheight}{16pt}
%\textheight = 20.5cm
%\hoffset
%\voffset
%\oddsidemargin = 18pt
%\topmargin = 0pt
%\headheight = 13.6pt
%\headsep = 25pt
%\textheight = 646pt
%\textwidth = 424pt
%\marginparsep = 11pt
%\marginparwidth = 54pt
%\footskip = 30pt
%\marginparpush = 5pt
%\hoffset = 0pt
%\voffset = 0pt
%\paperwidth = 597pt
%\paperheight = 845pt

%%%--- FANCYHDR ---%%%
\pagestyle{fancy}
\chead{}
\lhead{\aut}
\rhead{\tit}
\cfoot{}
\lfoot{\today}
\rfoot{\thepage}

%%%--- STANDARD TITLEPAGE ---%%% (Husk \maketitle)
\title{\tit}
\author{\aut}
\date{\today}

%%%--- MACRO INPUT ---%%%
%%%%%%%%%%%%%%%%%%%%%%%%%%%%%%%%%%%%%%%%%%%%%%%%%%%%%%%%
%%              Jacob Salomonsens Macros              %%
%%%%%%%%%%%%%%%%%%%%%%%%%%%%%%%%%%%%%%%%%%%%%%%%%%%%%%%%
%%%%%%%%%%%%%%%%%%%%%%%%%%%%%%%%%%%%%%%%%%%%%%%%%%%%%%%%
%%                       Layout                       %%
%%%%%%%%%%%%%%%%%%%%%%%%%%%%%%%%%%%%%%%%%%%%%%%%%%%%%%%%

\newcommand{\HRule}		{\rule{\linewidth}{0.5mm}}

\newcommand{\insfig}[4]{		% \insfig{path}{scale}{caption}{label}
\begin{figure}[H]
	\begin{center}
		\includegraphics[width=#2]{#1}
		\caption{#3}
		\label{#4}
	\end{center}	
\end{figure}
}

\newcommand{\instbl}[3]{		% \instbl{tabular}{caption}{label}
\begin{table}[htb]
	\begin{center}
		#1
		\caption{#2}
		\label{#3}
	\end{center}
\end{table}
}

%%%%%%%%%%%%%%%%%%%%%%%%%%%%%%%%%%%%%%%%%%%%%%%%%%%%%%%%
%%                        Math                        %%
%%%%%%%%%%%%%%%%%%%%%%%%%%%%%%%%%%%%%%%%%%%%%%%%%%%%%%%%

\newcommand{\maple}[2]{			% Til formatering af maple input, 1=kommando 2=output
\\\\\indent\texttt{[>#1}		% Brug \\ i output til at lave flere linier.
\begin{center}
\texttt{#2}
\end{center}
}

\newcommand{\matr}[4]{			% \matr{[}{]}{colums}{contents (\\ = change row)}
\left#1\!\begin{array}{#3}
#4
\end{array}\!\right#2
}

\newcommand{\Rm}[1]		{\mathrm{#1}}
\newcommand{\unit}[1]		{\ensuremath{\,\mathrm{#1}}}
\newcommand{\NN}		{\ensuremath{\mathbb N}}
\newcommand{\RR}		{\ensuremath{\mathbb R}}
\newcommand{\CC}		{\ensuremath{\mathbb C}}
\newcommand{\degree}		{\ensuremath{\,^{\circ}}}
\newcommand{\tii}[1]		{\ensuremath{\cdot10^{#1}}}
\newcommand{\vekt}[1]		{\ensuremath{\overrightarrow{#1}}}
\newcommand{\result}[1]		{\underline{\underline{#1}}}
\newcommand{\rd}		{\mathrm{d}}
\newcommand{\E}			{\mathrm{e}}
\newcommand{\I}			{\mathrm{i}}

%%%%%%%%%%%%%%%%%%%%%%%%%%%%%%%%%%%%%%%%%%%%%%%%%%%%%%%%
%%                      Physics                       %%
%%%%%%%%%%%%%%%%%%%%%%%%%%%%%%%%%%%%%%%%%%%%%%%%%%%%%%%%

\newcommand{\std}[1]		{#1^{\minuso}}
\newcommand{\stdm}[1]		{#1^{\minuso}_\Rm{m}}
\newcommand{\dstd}[2]		{\Delta_\mathrm{#1} #2^{\minuso}}

\newcommand{\dQ}		{\rd Q}
\newcommand{\dS}		{\rd S}
\newcommand{\dT}		{\rd T}
\newcommand{\dH}		{\rd H}
\newcommand{\dU}		{\rd U}
\newcommand{\dV}		{\rd V}
\newcommand{\dP}		{\rd p}
\newcommand{\dG}		{\rd G}
\newcommand{\dTau}		{\rd \tau}
\newcommand{\dN}		{\rd N}
\newcommand{\dn}		{\rd n}

\renewcommand{\DH}		{\Delta H}
\newcommand{\DG}		{\Delta G}
\newcommand{\DU}		{\Delta U}
\newcommand{\DS}		{\Delta S}
\newcommand{\DV}		{\Delta V}
\newcommand{\DT}		{\Delta T}
\newcommand{\DM}		{\Delta \mu}


\newcommand{\pstd}		{p^{\minuso}}
\newcommand{\sstd}		{S^{\minuso}}
\newcommand{\smstd}		{S_\Rm{m}^{\minuso}}

\newcommand{\cpstd}		{C_p^{\minuso}}
\newcommand{\cpmstd}		{C_{p,\Rm{m}}^{\minuso}}
\newcommand{\cpm}		{C_{p,\Rm{m}}}
\newcommand{\cp}		{C_{p}}

\newcommand{\cvstd}		{C_V^{\minuso}}
\newcommand{\cvmstd}		{C_{V,\Rm{m}}^{\minuso}}
\newcommand{\cvm}		{C_{V,\Rm{m}}}
\newcommand{\cv}		{C_{V}}

\newcommand{\Hm}		{H_\Rm{m}}
\newcommand{\kb}		{k_\Rm{B}}
\newcommand{\jpk}		{\unit{J}\unit{K^{-1}}}
\newcommand{\jm}		{\unit{J}\unit{mol^{-1}}}
\newcommand{\jkm}		{\unit{J}\unit{K^{-1}}\unit{mol^{-1}}}
\newcommand{\kj}		{\unit{kJ}}
\newcommand{\kjm}		{\unit{kJ}\unit{mol^{-1}}}
\newcommand{\gpm}		{\unit{g}\unit{mol^{-1}}}

\newcommand{\Tfus}		{T_\Rm{fus}}
\newcommand{\Tvap}		{T_\Rm{vap}}
\newcommand{\Tm}		{T_\Rm{m}}
\newcommand{\Tb}		{T_\Rm{b}}
\newcommand{\TRoom}		{298.15 \unit{K}}
\newcommand{\TMelt}		{273.15 \unit{K}}
\newcommand{\Kstd}		{\std{K}}

\newcommand{\Q}[2]		{#1 \unit{#2}}

%%%%%%%%%%%%%%%%%%%%%%%%%%%%%%%%%%%%%%%%%%%%%%%%%%%%%%%%
%%                     Chemistry                      %%
%%%%%%%%%%%%%%%%%%%%%%%%%%%%%%%%%%%%%%%%%%%%%%%%%%%%%%%%

% \reak{\chem{HCl}&+&\chem{NaOH}&$\rightarrow$&\chem{NaCl}&+&\chem{H_2O}}
\newcommand{\reak}[1]{
\begin{center}
	\begin{tabular}{ccccccccccccccccc}
		#1
	\end{tabular}
\end{center}
}

\newcommand{\chem}[1]	{\ensuremath{\mathrm{#1}}}
\newcommand{\lra}		{\longrightarrow}
\newcommand{\arr}		{\rightleftarrows}
\newcommand{\darr}		{\rightleftarrows}
\newcommand{\KS}		{\textrm{K}_s}
\newcommand{\KB}		{\textrm{K}_b}
\newcommand{\KV}		{\textrm{K}_v}
\newcommand{\mol}		{\textrm{mol}}
\newcommand{\g}			{\textrm{g}}
\newcommand{\gm}		{\frac{\g}{\mol}}
\newcommand{\M}			{\textrm{M}}
\newcommand{\konc}		{\frac{\mol}{\textrm{L}}}
\newcommand{\pH}		{\textrm{pH}}
\newcommand{\pOH}		{\textrm{pOH}}

%%%%%%%%%%%%%%%%%%%%%%%%%%%%%%%%%%%%%%%%%%%%%%%%%%%%%%%%%%%%%%%%%%%%%%%%
%% Chemical species
%%%%%%%%%%%%%%%%%%%%%%%%%%%%%%%%%%%%%%%%%%%%%%%%%%%%%%%%%%%%%%%%%%%%%%%%
\newcommand{\ammnitrats}{\chem{NH_4NO_3(s)}}
\newcommand{\benzen}    {\chem{C_6H_6}}
\newcommand{\borto}     {\chem{B_2}}
\newcommand{\bor}       {\chem{B}}
\newcommand{\brmal}     {\chem{BrCH(COOH)_2}}
\newcommand{\brma}      {\chem{BrMA}}
\newcommand{\brmsf}     {\chem{{\sf [BrMA]}}}
\newcommand{\bromat}    {\chem{BrO_3^-}}
\newcommand{\bromid}    {\chem{Br^-}}
\newcommand{\brom}      {\chem{Br^-}}
\newcommand{\brsyr}     {\chem{HBrO_2}}
\newcommand{\brto}      {\chem{Br_2}}
\newcommand{\calc}      {\chem{Ca^{2+}}}
\newcommand{\cefsf}     {\chem{{\sf [Ce^{4+}]}}}
\newcommand{\cef}       {\chem{Ce^{4+}}}
\newcommand{\cet}       {\chem{Ce^{3+}}}
\newcommand{\chloridc}  {\chem{[Cl^-]}}
\newcommand{\clto}      {\chem{Cl_2}}
\newcommand{\cltog}     {\chem{Cl_2(g)}}
\newcommand{\carbon}    {\chem{C}}
\newcommand{\carbongr}  {\chem{C(grafit)}}
\newcommand{\chlorid}   {\chem{Cl^-}}
\newcommand{\coto}      {\chem{CO_2}}
\newcommand{\cotos}     {\chem{CO_2(s)}}
\newcommand{\cotol}     {\chem{CO_2(l)}}
\newcommand{\cotog}     {\chem{CO_2(g)}}
\newcommand{\co}        {\chem{CO}}
\newcommand{\fe}        {\chem{Fe}}

\newcommand{\hbro}      {\chem{HBrO}}
\newcommand{\hoclc}     {\chem{[HOCl]}}
\newcommand{\hocl}      {\chem{HOCl}}
\newcommand{\hoic}      {\chem{[HOI]}}
\newcommand{\hoi}       {\chem{HOI}}
\newcommand{\hpc}       {\chem{[H^+]}}
\newcommand{\hp}        {\chem{H^+}}

\newcommand{\htooc}     {\chem{[H_20]}}
\newcommand{\htoo}      {\chem{H_2O}}
\newcommand{\htoog}     {\chem{H_2O(g)}}
\newcommand{\htool}     {\chem{H_2O(l)}}
\newcommand{\htoos}     {\chem{H_2O(s)}}
\newcommand{\hcl}       {\chem{HCl}}
\newcommand{\hclg}      {\chem{HCl(g)}}

\newcommand{\hto}       {\chem{H_2}}
\newcommand{\htog}      {\chem{H_2(g)}}
\newcommand{\htol}      {\chem{H_2(l)}}
\newcommand{\htos}      {\chem{H_2(s)}}

\newcommand{\iodidc}    {\chem{[I^-]}}
\newcommand{\iodid}     {\chem{I^-}}
\newcommand{\ipt}       {\chem{IP_3}}
\newcommand{\mal}       {\chem{CH_2(COOH)_2}}
\newcommand{\ma}        {\chem{MA}}
\newcommand{\ethan}     {\chem{C_2H_6}}
\newcommand{\ethang}    {\chem{C_2H_6(g)}}
\newcommand{\methan}    {\chem{CH_4}}
\newcommand{\methang}   {\chem{CH_4(g)}}
\newcommand{\mn}        {\chem{Mn}}
\newcommand{\naphtalen} {\chem{C_{10}H_8}}
\newcommand{\nhtre}     {\chem{NH_3}}
\newcommand{\nhtreg}    {\chem{NH_3(g)}}
\newcommand{\ntoog}     {\chem{N_2O(g)}}
\newcommand{\nto}       {\chem{N_2}}
\newcommand{\ntog}      {\chem{N_2}}
\newcommand{\oclmc}     {\chem{[OCl^-]}}
\newcommand{\oclm}      {\chem{OCl^-}}
\newcommand{\ohmc}      {\chem{[OH^-]}}
\newcommand{\ohm}       {\chem{OH^-}}
\newcommand{\oimc}      {\chem{[OI^-]}}
\newcommand{\oim}       {\chem{OI^-}}

\newcommand{\oto}       {\chem{O_2}}
\newcommand{\otog}      {\chem{O_2(g)}}
\newcommand{\ozon}      {\chem{O_3}}
\newcommand{\ozong}     {\chem{O_3(g)}}

\newcommand{\phenol}    {\chem{C_6H_5OH}}
\newcommand{\rut}       {\chem{Ru}}
\newcommand{\sotog}     {\chem{SO_2(g)}}
\newcommand{\sotol}     {\chem{SO_2(l)}}
\newcommand{\sotos}     {\chem{SO_2(s)}}
\newcommand{\soto}      {\chem{SO_2}}

\newcommand{\sns}       {\chem{Sn(s)}}
\newcommand{\snotos}    {\chem{SnO_2(s)}}

%%%%%%%%%%%%%%%%%%%%%%%%%%%%%%%%%%%%%%%%%%%%%%%%%%%%%%%%%%%%%%%%%%%%%%%%
%% References
%%%%%%%%%%%%%%%%%%%%%%%%%%%%%%%%%%%%%%%%%%%%%%%%%%%%%%%%%%%%%%%%%%%%%%%%

\newcommand{\hpref}[2]{\hyperref[#2]{#1 \ref{#2}}}

%%%%%%%%%%%%%%%%%%%%%%%%%%%%%%%%%%%%%%%%%%%%%%%%%%%%%%%%%%%%%%%%%%%%%%%%
%% Custom environments
%%%%%%%%%%%%%%%%%%%%%%%%%%%%%%%%%%%%%%%%%%%%%%%%%%%%%%%%%%%%%%%%%%%%%%%%

\newfloat{grammar}{thp}{lop}
\floatname{grammar}{Grammar}
%\usepackage[danish]{babel}				% Dansk orddeling osv.
%\usepackage[fixlanguage]{babelbib}		% Sprogpakke til BibTeX
%\selectbiblanguage{danish}				% Sprogvalg til BibTeX

%%%--- LISTISTINGS SETUP ---%%%
\lstset{ %
language=ML,							% choose the language of the code
basicstyle=\footnotesize,				% the size of the fonts that are used for the code
numbers=left,							% where to put the line-numbers
numberstyle=\footnotesize,				% the size of the fonts that are used for the line-numbers
stepnumber=2,							% the step between two line-numbers. If it's 1 each line will be numbered
numbersep=5pt,							% how far the line-numbers are from the code
backgroundcolor=\color{white},			% choose the background color. You must add \usepackage{color}
showspaces=false,						% show spaces adding particular underscores
showstringspaces=false,					% underline spaces within strings
showtabs=false,							% show tabs within strings adding particular underscores
frame=single,							% adds a frame around the code
tabsize=2,								% sets default tabsize to 2 spaces
captionpos=b,							% sets the caption-position to bottom
breaklines=true,						% sets automatic line breaking
breakatwhitespace=false,				% sets if automatic breaks should only happen at whitespace
escapeinside={\%*}{*)}					% if you want to add a comment within your code
}

%%%--- DOKUMENT STARTER HER ---%%%
\begin{document}

%%%--- Title ---%%%%
\begin{titlepage}
\HRule
\begin{center}\huge{\bfseries{\tit}}\end{center}
\HRule
%\\[2.0cm]
\begin{center}
\aut
\\[0.5cm]
\mail
\\[0.5cm]
\subj
\end{center}
\vfill
\begin{center}\today\end{center}
\end{titlepage}

%%%--- Indhold ---%%%
%\tableofcontents
%\newpage

%%%--- Tekst ---%%%

\section{Intro}
ITIL handler om kunders forventining til service. ITIL består af 5 faser der varetager en applikations livscyklus:

\begin{itemize}
\item Continuous improvement
\item Operation
\item Transition
\item Design
\item Strategy
\end{itemize}

Produkt kontra service: Ved service er det leverandørens ansvar at sørge for kvaliteten. Modsat ved et produkt (ex. mælk i Netto) er det kundens ansvar at sørge for at kvaliteten lever op til forventning.

ITIL søger at klarlægge behovet. Forretningsprocesser som skal understøttes.

Alle services kan granuleres til underlinggende services og produkter, inklusiv hardware og software.

SLA: kontrakt mellem service leverandør om kunde. Kunden ser kun selve service.

\section{Hvad er en service}

Ved CSM løsning skal man kunne ringe ud og ind. Sende og modtage mail/SMS.
\begin{itemize}
\item Utility: Hvad kan servicen rent funktionelt.
\item Warranty: Tilgænglighed, kapacitet, sikkerhed, kontinuitet.
\end{itemize}

SLA består af utility og warranty som er målbart. Alt der ikke kan måles SKAL ikke inkluderes i SLA.

Forretningsproces: Det kunderne gør som de har brug for hjælp til fra os.

ITIL service management består af:
\begin{center}
\begin{tabular}{l||c|c|c|c}
kunder & SKAT & ASK & Top DK. & DEAS \\
Forretnings process. & Toldsystemer & Kundeservice & Administration & \\
Ansvaret skiller her & SLA & SLA & SLA & SLA\\
IT-Services & CSM & Desktop mgmt. & BPM & Faktura scanning \\
ITIL er i dette lag & & &  & \\
Komponenter & HW & APP & Netv. & Data
\end{tabular}
\end{center}


\section{Service Management}
\subsection{ITIL som rammeværk}
ISO 20000 er standard for it service management. ITIL er et rammeværk hvilket vil sige det kan tilpasses til virksomhedens virkelighed.
Itil er en løbende process som udvikles kontinuert. ITIL versioner er forskellige fra hindanden.

\begin{itemize}
\item Utility
\item Warranty
\end{itemize}

Utility består af performance eller fjernelse af begrænsninger.

De fire warranty elementer
\begin{itemize}
\item Availability (Alle relationer mellem en service skal undersøges i forbindelse med avaliability).
\item Capacity (Svartider, capacity påvirker availability)
\item Continuity (Katastrofeberedskab, back-up planer)
\item Security (Beskytning af data)
\end{itemize}

Alle elementer kan måles og sættes ind i en SLA

Organisatoriske capabilities: evner i virksomheden, personers reele funktioner.

It service som består af utility og warranty. Kunden ser oftest mest på utility.

People, Processes, Products, Partners

Kunden har selv ansvar for at deres forretningsprocesser udføres af servicen, som it-organisation har vi kune indflydelse på dette. Jo mere man ved om hvad forretningen vil opnå jo bedre er slutproduktet (outcome).

\subsection{Iteressenter}
Funktion har en særlig betydning i ITIL: Et team eller en gruppe af medarbejder som har værktøjer til rådighed eller andre ressourcer de anvender for at udføre processer eller aktiviteteter.

\subsection{Assets}
Assets er alt det man bruger til at levere en service. Opdeles i capabilities og ressourcer. 

Alle ressourcer kan måles på en eller anden facon.

Capabilities er ikke håndgribelige som Ressourcer. Processer i diagrammet, evnen til at følge processer. Capabilities styrer og koordinerer ressourcer gennem processer.

\subsection{Procesmodellen}
Der er altid noget der trigger en proces.

-Check op på raki matricen.

Så snart man har klarlagt/tegnet processen vil man kunne finde målepunkter.

Formidlere (enablers), gør processen mulig. Processkontrol styrer processen og sørger for at den følges.

Feedback kan komme i form af forbedringsønsker eller feedback på målinger.

Incident producerer workaround og kan trigger problem.

Change producerer en behandlet change request (ja/nej).

Der kan være mange servicejere men der kan kun være en ejer per service.

Silodannelse, en afdeling afsondrer sig fra resten af organisationen, højt specialiseret. ITIL higer efter at nedbryde siloer.

Accountable: man kan stilles til regnskab for det arbejde der bliver udført.

Der eksisterer mange til mange forhold mellem services og processer.

Service desk eller team lead vil ofte være process manager, som sørger for at service desken har sin normale daglige operation. Kan sammelignes med en værkfører.

Process ejeren har ikke indblik i den daglige operation, står uden for processen, men definerer selve process med målepunkter.

Matriceorganisation dækker over at man ikke skaber siloer.

\subsection{Automatisering}
Hvad skal værktøj dække af formål. Man skal kende processen før man kan finde værktøj. 

Change hører til i transitions værktøj da den er en del af udrulning og sætter ting i gang.

OLA intern SLA.

\section{Service Strategy}
Tag udgangspunkt i forretningens slutprodukt, resultat, outcome, udbytte.
Denne fase handler om at finde frem til hvad kunden skal bruge.

Virksomhedsstrategi, et overordnet mål for hele virksomheden.

IT-Strategi, hvilke tekniske mål sættes for organisationen. Eksempel, brug MS produkter.

Service strategi bliver påvirket af alle strategier.

Værdi er ikke altid at definere, da kunden er med til at definere store dele af dette. Man kan definere Utility og Warranty.
Kender man kundes preferencer kan man få utility og warranty til at betyde mindre. 

Tre forskellige typer services

\begin{itemize}
\item Core services (Basale services som service normalt indeholder, direkte understøtter forretningsprocessen)(Klipning)
\item Enabling services (services som kunden sjældent ser)(slipning af saks)
\item Enhancing services (Klare services som ses som ekstra der giver en wow faktor)(Ugeblade, kaffe)(bliver efter tid til core services)
\end{itemize}

\subsection{Serviceleverandør typer}
Intern serviceleverandør, leverer til specifik afdeling.

Shared service unit, leverer til flere afdelinger.

Ekstern serviceleverandør, leverer til kunde, outsourcing.

\subsection{Governance}
Kontrol og beslutningstagen. Jo mere kontrol og styring jo sværere bliver det at være effektiv og producere. Der skal skabes en balance mellem kontrol styring kontra frihed og handlemulighed. Regler overfor beslutninger.

Orderliness -- Performance.

Udlevelse af strategier og retningslinier med fokus på balance mellem ovenstående, uden at give for meget køb på performance.

ITIL fokuserer mest på værdiskabelse og ressourceudnyttelse.

\subsection{Risk management}
Risk management er en bred disciplin som benyttes i alle faser af ITIL.

\begin{itemize}
\item identificere
\item analysere
\item håndter
\end{itemize}

Vurdering af sandsynlighed og impact. Hvad koster counter-measures. Dette giver en prioritering af hvilke risici der skal håndteres først.

\subsection{Service portfolio management}
At sørge for at have de services der løser kundens behov.

Hvis kundernes behov flytter sig skal spm sørge for at services også flytter sig.

I et CMS findes information om service portfolio på alle niveauer.

Strategisk investering i leverance apparatet, er investeringer organisationen gør som kunden ikke er involveret i, evt en nødvendig opgradering.

Kommisorium, der skal være kunder som kan levere penge til en service.

Portfolio management:
\begin{itemize}
\item Definerer
\item Analyserer
\item Godkender (Business case opstår her)
\item Charter (penge)
\end{itemize}

Leder til service design fasen:
\begin{itemize}
\item Udvikling (Service design)
\item Idriftsættelse (Service transition)
\item Drift (Service operation)(Service catalog)
\item Udfasning (Service transition) (Portfolion management udføres også her)
\end{itemize}

Som til sidst leder til udfasning af service.

\subsection{Financial management}

\subsection{Business case}
Cost benefit vurdering af krav.


\subsection{Business Relationship management}
Strategisk: mulighed for mersalg, både eksternt til kunder men også til internt brug. Dette giver øget mersalg og effektivitet internt men viser sig også eksternt. Vær tæt på forretningen. (business relationship manager med produkt katalog)

Taktisk: er mere en SDM, Service Level manager. 

Operationelt: Service desk, direkte kontakt.

Product owner er tættest på service ejer i ITIL.

Kundens strategiske del kan være langt fra selve brugerne.

\subsection{Demand management}
Der kan forekomme udsving i kundernes behov. Dette kan enten koste at dække da der uden for spidsbelastning kan opstå spild. Man kan dimensionere service efter normal kørsel, men dette leder til utilfredse kunder.

I stedet for ovenstående kan man motivere kunder til at modificere deres adfærd ved enten pisk eller gulerod.

Demand management, styring af efterspørgslen
Capacity management, styring af selve kapaciteten.


\subsection{Opsummering}

Kunde og IT ledelse aftaler SLA. IT Ledelse sætter udvikling i gang. Udvikling afleverer til brugere. Brugerne melder krav og behov til kunden. Hvis udvikling kommunikerer med brugere kan det give bedre løsning. Forventningens vej.

Udbud -> Service katalog.

Efterspørgsel -> Service pipeline.

Outcome


\section{Service Design}
Hvordan får vi lavet hvad kunden har behov for. Overordnede ting som peger i retning af hvad kunde har brug for. I designfasen gælder det om ikke at overlade noget til tilfældigheder. Derfor er det et spørgsmål om at være opmærksom på alle detaljer.

Man ser på service landskabet først. Man gør det i den rigtige rækkefølge, alltså fra kundens synspunkt.

Design fasen indeholde også design af selve processerne.

I design fasen laves der kontrakter på samme måde som i service strategy laves SLA.

De 4 p'er (People, processes, product, partners) skal indgå i designfasen.

Hovedaspekter af service design

\begin{itemize}
\item Serviceløsninger (se overordnet på outcome)
\item Management information systemer - portfolio ()
\item Teknologi og management (server/klient, data arkitektur, produkt arkitektur, monitorering)
\item Processer (Skal nuværende processer udvides, kun ITIL)
\item Målesystemer (SLA og OLA skal der måles efter om opfyldes)
\end{itemize}

\subsection{Service løsninger}
Early life support opstår når en service overgår fra transition til operation. Her omformuleres SLA hvis service ikke lever op til kravene. Service level management.


























\section{Forkortelser}
\begin{itemize}
\item OLA - Operational Level Agreement - Intern aftale som skal støtte op om overholdelse af SLA. Procent oppetid på server, tabte netværkspakker.
\item SLA - Service Level Agreement - Kontrakt mellem service leverandør og kunde. Utility/Warranty. Svartid på kundehendvendelser, benahdlingstid på ændringsønsker. Kan ændre sig over tid.
\item SAC - Service Acceptance Criteria - Overordnede kriterier som en service skal kunne opfylde i hele levetiden.
\item SLR - Service Level Requirement - 
\item SDP - Service Design Package - Den "perfekte" beskrivelse af hvordan service skal implementeres. Perfekt blueprint. Fuldstændig specifikation af alle parametre for service.
\item RFC - Request For Change
\end{itemize}





%%%---BibTeX ---%%%
%(Compile: 1 x LaTeX 1 x Bibtex 2 x LaTex)%
%\newpage
%\bibliographystyle{plain}
%\bibliography{/Tex/Library}

\end{document}
